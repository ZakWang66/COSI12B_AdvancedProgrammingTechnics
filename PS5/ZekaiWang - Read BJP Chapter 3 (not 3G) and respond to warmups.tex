\documentclass[12pt]{article}
\usepackage{geometry}
\usepackage{fontspec}
\geometry{letterpaper,left=3.18cm,right=3.18cm,top=2.54cm,bottom=2.54cm}

\begin{document}
\setmainfont{Times New Roman}
\noindent
\textbf{Zekai Wang - 1/28: [PARTICIPATION] \\Read BJP Chapter 1 and 2 and respond to warmups\\}
\\[0pt]
1. What is the difference between a formal and an actual parameter? Use your own words.\\
\\[0pt]
The actual parameter is the parameter passing to the method when calling them.\\
The formal parameter is the parameter defined when writing a method, it can be called in the method and the value depends on the actual parameter passed in.\\
\\[0pt]
2. In what way is a String object like an array, and in what way different?\\
\\[0pt]
A String object in Java is just like a char array. (I would say it is in C/C++...) However, since a String in Java is packaged as an Object, we can't use 
the syntax of calling a char element in a char array to get a character in a particular position. Instead, to a String let's say someString, we call 
someString.charAt(i) to get the character at index i. There might be a char array field in the implementation of the String class, but since it's an 
Object, we need to follow the rules of interacting with Object when using it. This also provides us a lot of useful methods such as "split", "subString", 
which makes it easier for us to process the Strings.\\
\\[0pt]
3. Methods often return information to the caller (not always!). What are all the different ways you can think up for a method to return info the the caller? Use your own words!\\
\\[0pt]
1. If a method's return type is not void, it can return am Object or a primitive type value that contains information to its caller.\\
2. A method can modify fields declared outside of it, which can be seen not only by itself, but also by some other parts of the program, including its caller...\\
3. A method can modify an object passed in by parameter, which is a kind of reference. The modification will be done on the particular object, since the parameter 
   is actually a pointer pointing to it.\\
\\[0pt]
4. List two or three concepts from the readings that you are still unsure or confused about\\
\\[0pt]
One thing confuses me is that when I'm reading the java official documentations, I find it seems that Java designers defined some kind of a Class named as some mark 
ending with "[]", this makes me think that arrays in java is probably also a kind of Object with special syntax. More over, I even think about more Java syntax are 
probably written by special Objects, such as "instanceof" ? (even "if","while","for"...? I don't know...). It is really interesting how they make this works...
\end{document}

